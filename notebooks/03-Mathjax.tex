
% Default to the notebook output style

    


% Inherit from the specified cell style.




    
\documentclass[11pt]{article}

    
    
    \usepackage[T1]{fontenc}
    % Nicer default font (+ math font) than Computer Modern for most use cases
    \usepackage{mathpazo}

    % Basic figure setup, for now with no caption control since it's done
    % automatically by Pandoc (which extracts ![](path) syntax from Markdown).
    \usepackage{graphicx}
    % We will generate all images so they have a width \maxwidth. This means
    % that they will get their normal width if they fit onto the page, but
    % are scaled down if they would overflow the margins.
    \makeatletter
    \def\maxwidth{\ifdim\Gin@nat@width>\linewidth\linewidth
    \else\Gin@nat@width\fi}
    \makeatother
    \let\Oldincludegraphics\includegraphics
    % Set max figure width to be 80% of text width, for now hardcoded.
    \renewcommand{\includegraphics}[1]{\Oldincludegraphics[width=.8\maxwidth]{#1}}
    % Ensure that by default, figures have no caption (until we provide a
    % proper Figure object with a Caption API and a way to capture that
    % in the conversion process - todo).
    \usepackage{caption}
    \DeclareCaptionLabelFormat{nolabel}{}
    \captionsetup{labelformat=nolabel}

    \usepackage{adjustbox} % Used to constrain images to a maximum size 
    \usepackage{xcolor} % Allow colors to be defined
    \usepackage{enumerate} % Needed for markdown enumerations to work
    \usepackage{geometry} % Used to adjust the document margins
    \usepackage{amsmath} % Equations
    \usepackage{amssymb} % Equations
    \usepackage{textcomp} % defines textquotesingle
    % Hack from http://tex.stackexchange.com/a/47451/13684:
    \AtBeginDocument{%
        \def\PYZsq{\textquotesingle}% Upright quotes in Pygmentized code
    }
    \usepackage{upquote} % Upright quotes for verbatim code
    \usepackage{eurosym} % defines \euro
    \usepackage[mathletters]{ucs} % Extended unicode (utf-8) support
    \usepackage[utf8x]{inputenc} % Allow utf-8 characters in the tex document
    \usepackage{fancyvrb} % verbatim replacement that allows latex
    \usepackage{grffile} % extends the file name processing of package graphics 
                         % to support a larger range 
    % The hyperref package gives us a pdf with properly built
    % internal navigation ('pdf bookmarks' for the table of contents,
    % internal cross-reference links, web links for URLs, etc.)
    \usepackage{hyperref}
    \usepackage{longtable} % longtable support required by pandoc >1.10
    \usepackage{booktabs}  % table support for pandoc > 1.12.2
    \usepackage[inline]{enumitem} % IRkernel/repr support (it uses the enumerate* environment)
    \usepackage[normalem]{ulem} % ulem is needed to support strikethroughs (\sout)
                                % normalem makes italics be italics, not underlines
    

    
    
    % Colors for the hyperref package
    \definecolor{urlcolor}{rgb}{0,.145,.698}
    \definecolor{linkcolor}{rgb}{.71,0.21,0.01}
    \definecolor{citecolor}{rgb}{.12,.54,.11}

    % ANSI colors
    \definecolor{ansi-black}{HTML}{3E424D}
    \definecolor{ansi-black-intense}{HTML}{282C36}
    \definecolor{ansi-red}{HTML}{E75C58}
    \definecolor{ansi-red-intense}{HTML}{B22B31}
    \definecolor{ansi-green}{HTML}{00A250}
    \definecolor{ansi-green-intense}{HTML}{007427}
    \definecolor{ansi-yellow}{HTML}{DDB62B}
    \definecolor{ansi-yellow-intense}{HTML}{B27D12}
    \definecolor{ansi-blue}{HTML}{208FFB}
    \definecolor{ansi-blue-intense}{HTML}{0065CA}
    \definecolor{ansi-magenta}{HTML}{D160C4}
    \definecolor{ansi-magenta-intense}{HTML}{A03196}
    \definecolor{ansi-cyan}{HTML}{60C6C8}
    \definecolor{ansi-cyan-intense}{HTML}{258F8F}
    \definecolor{ansi-white}{HTML}{C5C1B4}
    \definecolor{ansi-white-intense}{HTML}{A1A6B2}

    % commands and environments needed by pandoc snippets
    % extracted from the output of `pandoc -s`
    \providecommand{\tightlist}{%
      \setlength{\itemsep}{0pt}\setlength{\parskip}{0pt}}
    \DefineVerbatimEnvironment{Highlighting}{Verbatim}{commandchars=\\\{\}}
    % Add ',fontsize=\small' for more characters per line
    \newenvironment{Shaded}{}{}
    \newcommand{\KeywordTok}[1]{\textcolor[rgb]{0.00,0.44,0.13}{\textbf{{#1}}}}
    \newcommand{\DataTypeTok}[1]{\textcolor[rgb]{0.56,0.13,0.00}{{#1}}}
    \newcommand{\DecValTok}[1]{\textcolor[rgb]{0.25,0.63,0.44}{{#1}}}
    \newcommand{\BaseNTok}[1]{\textcolor[rgb]{0.25,0.63,0.44}{{#1}}}
    \newcommand{\FloatTok}[1]{\textcolor[rgb]{0.25,0.63,0.44}{{#1}}}
    \newcommand{\CharTok}[1]{\textcolor[rgb]{0.25,0.44,0.63}{{#1}}}
    \newcommand{\StringTok}[1]{\textcolor[rgb]{0.25,0.44,0.63}{{#1}}}
    \newcommand{\CommentTok}[1]{\textcolor[rgb]{0.38,0.63,0.69}{\textit{{#1}}}}
    \newcommand{\OtherTok}[1]{\textcolor[rgb]{0.00,0.44,0.13}{{#1}}}
    \newcommand{\AlertTok}[1]{\textcolor[rgb]{1.00,0.00,0.00}{\textbf{{#1}}}}
    \newcommand{\FunctionTok}[1]{\textcolor[rgb]{0.02,0.16,0.49}{{#1}}}
    \newcommand{\RegionMarkerTok}[1]{{#1}}
    \newcommand{\ErrorTok}[1]{\textcolor[rgb]{1.00,0.00,0.00}{\textbf{{#1}}}}
    \newcommand{\NormalTok}[1]{{#1}}
    
    % Additional commands for more recent versions of Pandoc
    \newcommand{\ConstantTok}[1]{\textcolor[rgb]{0.53,0.00,0.00}{{#1}}}
    \newcommand{\SpecialCharTok}[1]{\textcolor[rgb]{0.25,0.44,0.63}{{#1}}}
    \newcommand{\VerbatimStringTok}[1]{\textcolor[rgb]{0.25,0.44,0.63}{{#1}}}
    \newcommand{\SpecialStringTok}[1]{\textcolor[rgb]{0.73,0.40,0.53}{{#1}}}
    \newcommand{\ImportTok}[1]{{#1}}
    \newcommand{\DocumentationTok}[1]{\textcolor[rgb]{0.73,0.13,0.13}{\textit{{#1}}}}
    \newcommand{\AnnotationTok}[1]{\textcolor[rgb]{0.38,0.63,0.69}{\textbf{\textit{{#1}}}}}
    \newcommand{\CommentVarTok}[1]{\textcolor[rgb]{0.38,0.63,0.69}{\textbf{\textit{{#1}}}}}
    \newcommand{\VariableTok}[1]{\textcolor[rgb]{0.10,0.09,0.49}{{#1}}}
    \newcommand{\ControlFlowTok}[1]{\textcolor[rgb]{0.00,0.44,0.13}{\textbf{{#1}}}}
    \newcommand{\OperatorTok}[1]{\textcolor[rgb]{0.40,0.40,0.40}{{#1}}}
    \newcommand{\BuiltInTok}[1]{{#1}}
    \newcommand{\ExtensionTok}[1]{{#1}}
    \newcommand{\PreprocessorTok}[1]{\textcolor[rgb]{0.74,0.48,0.00}{{#1}}}
    \newcommand{\AttributeTok}[1]{\textcolor[rgb]{0.49,0.56,0.16}{{#1}}}
    \newcommand{\InformationTok}[1]{\textcolor[rgb]{0.38,0.63,0.69}{\textbf{\textit{{#1}}}}}
    \newcommand{\WarningTok}[1]{\textcolor[rgb]{0.38,0.63,0.69}{\textbf{\textit{{#1}}}}}
    
    
    % Define a nice break command that doesn't care if a line doesn't already
    % exist.
    \def\br{\hspace*{\fill} \\* }
    % Math Jax compatability definitions
    \def\gt{>}
    \def\lt{<}
    % Document parameters
    \title{03-Mathjax}
    
    
    

    % Pygments definitions
    
\makeatletter
\def\PY@reset{\let\PY@it=\relax \let\PY@bf=\relax%
    \let\PY@ul=\relax \let\PY@tc=\relax%
    \let\PY@bc=\relax \let\PY@ff=\relax}
\def\PY@tok#1{\csname PY@tok@#1\endcsname}
\def\PY@toks#1+{\ifx\relax#1\empty\else%
    \PY@tok{#1}\expandafter\PY@toks\fi}
\def\PY@do#1{\PY@bc{\PY@tc{\PY@ul{%
    \PY@it{\PY@bf{\PY@ff{#1}}}}}}}
\def\PY#1#2{\PY@reset\PY@toks#1+\relax+\PY@do{#2}}

\expandafter\def\csname PY@tok@s2\endcsname{\def\PY@tc##1{\textcolor[rgb]{0.73,0.13,0.13}{##1}}}
\expandafter\def\csname PY@tok@gh\endcsname{\let\PY@bf=\textbf\def\PY@tc##1{\textcolor[rgb]{0.00,0.00,0.50}{##1}}}
\expandafter\def\csname PY@tok@ge\endcsname{\let\PY@it=\textit}
\expandafter\def\csname PY@tok@gs\endcsname{\let\PY@bf=\textbf}
\expandafter\def\csname PY@tok@mb\endcsname{\def\PY@tc##1{\textcolor[rgb]{0.40,0.40,0.40}{##1}}}
\expandafter\def\csname PY@tok@nf\endcsname{\def\PY@tc##1{\textcolor[rgb]{0.00,0.00,1.00}{##1}}}
\expandafter\def\csname PY@tok@c\endcsname{\let\PY@it=\textit\def\PY@tc##1{\textcolor[rgb]{0.25,0.50,0.50}{##1}}}
\expandafter\def\csname PY@tok@vi\endcsname{\def\PY@tc##1{\textcolor[rgb]{0.10,0.09,0.49}{##1}}}
\expandafter\def\csname PY@tok@o\endcsname{\def\PY@tc##1{\textcolor[rgb]{0.40,0.40,0.40}{##1}}}
\expandafter\def\csname PY@tok@ow\endcsname{\let\PY@bf=\textbf\def\PY@tc##1{\textcolor[rgb]{0.67,0.13,1.00}{##1}}}
\expandafter\def\csname PY@tok@s\endcsname{\def\PY@tc##1{\textcolor[rgb]{0.73,0.13,0.13}{##1}}}
\expandafter\def\csname PY@tok@nc\endcsname{\let\PY@bf=\textbf\def\PY@tc##1{\textcolor[rgb]{0.00,0.00,1.00}{##1}}}
\expandafter\def\csname PY@tok@cm\endcsname{\let\PY@it=\textit\def\PY@tc##1{\textcolor[rgb]{0.25,0.50,0.50}{##1}}}
\expandafter\def\csname PY@tok@na\endcsname{\def\PY@tc##1{\textcolor[rgb]{0.49,0.56,0.16}{##1}}}
\expandafter\def\csname PY@tok@gp\endcsname{\let\PY@bf=\textbf\def\PY@tc##1{\textcolor[rgb]{0.00,0.00,0.50}{##1}}}
\expandafter\def\csname PY@tok@ne\endcsname{\let\PY@bf=\textbf\def\PY@tc##1{\textcolor[rgb]{0.82,0.25,0.23}{##1}}}
\expandafter\def\csname PY@tok@cp\endcsname{\def\PY@tc##1{\textcolor[rgb]{0.74,0.48,0.00}{##1}}}
\expandafter\def\csname PY@tok@sd\endcsname{\let\PY@it=\textit\def\PY@tc##1{\textcolor[rgb]{0.73,0.13,0.13}{##1}}}
\expandafter\def\csname PY@tok@il\endcsname{\def\PY@tc##1{\textcolor[rgb]{0.40,0.40,0.40}{##1}}}
\expandafter\def\csname PY@tok@no\endcsname{\def\PY@tc##1{\textcolor[rgb]{0.53,0.00,0.00}{##1}}}
\expandafter\def\csname PY@tok@k\endcsname{\let\PY@bf=\textbf\def\PY@tc##1{\textcolor[rgb]{0.00,0.50,0.00}{##1}}}
\expandafter\def\csname PY@tok@se\endcsname{\let\PY@bf=\textbf\def\PY@tc##1{\textcolor[rgb]{0.73,0.40,0.13}{##1}}}
\expandafter\def\csname PY@tok@nl\endcsname{\def\PY@tc##1{\textcolor[rgb]{0.63,0.63,0.00}{##1}}}
\expandafter\def\csname PY@tok@sh\endcsname{\def\PY@tc##1{\textcolor[rgb]{0.73,0.13,0.13}{##1}}}
\expandafter\def\csname PY@tok@sc\endcsname{\def\PY@tc##1{\textcolor[rgb]{0.73,0.13,0.13}{##1}}}
\expandafter\def\csname PY@tok@nd\endcsname{\def\PY@tc##1{\textcolor[rgb]{0.67,0.13,1.00}{##1}}}
\expandafter\def\csname PY@tok@nv\endcsname{\def\PY@tc##1{\textcolor[rgb]{0.10,0.09,0.49}{##1}}}
\expandafter\def\csname PY@tok@gr\endcsname{\def\PY@tc##1{\textcolor[rgb]{1.00,0.00,0.00}{##1}}}
\expandafter\def\csname PY@tok@kp\endcsname{\def\PY@tc##1{\textcolor[rgb]{0.00,0.50,0.00}{##1}}}
\expandafter\def\csname PY@tok@kc\endcsname{\let\PY@bf=\textbf\def\PY@tc##1{\textcolor[rgb]{0.00,0.50,0.00}{##1}}}
\expandafter\def\csname PY@tok@gt\endcsname{\def\PY@tc##1{\textcolor[rgb]{0.00,0.27,0.87}{##1}}}
\expandafter\def\csname PY@tok@fm\endcsname{\def\PY@tc##1{\textcolor[rgb]{0.00,0.00,1.00}{##1}}}
\expandafter\def\csname PY@tok@mi\endcsname{\def\PY@tc##1{\textcolor[rgb]{0.40,0.40,0.40}{##1}}}
\expandafter\def\csname PY@tok@bp\endcsname{\def\PY@tc##1{\textcolor[rgb]{0.00,0.50,0.00}{##1}}}
\expandafter\def\csname PY@tok@err\endcsname{\def\PY@bc##1{\setlength{\fboxsep}{0pt}\fcolorbox[rgb]{1.00,0.00,0.00}{1,1,1}{\strut ##1}}}
\expandafter\def\csname PY@tok@si\endcsname{\let\PY@bf=\textbf\def\PY@tc##1{\textcolor[rgb]{0.73,0.40,0.53}{##1}}}
\expandafter\def\csname PY@tok@sx\endcsname{\def\PY@tc##1{\textcolor[rgb]{0.00,0.50,0.00}{##1}}}
\expandafter\def\csname PY@tok@vm\endcsname{\def\PY@tc##1{\textcolor[rgb]{0.10,0.09,0.49}{##1}}}
\expandafter\def\csname PY@tok@kr\endcsname{\let\PY@bf=\textbf\def\PY@tc##1{\textcolor[rgb]{0.00,0.50,0.00}{##1}}}
\expandafter\def\csname PY@tok@m\endcsname{\def\PY@tc##1{\textcolor[rgb]{0.40,0.40,0.40}{##1}}}
\expandafter\def\csname PY@tok@c1\endcsname{\let\PY@it=\textit\def\PY@tc##1{\textcolor[rgb]{0.25,0.50,0.50}{##1}}}
\expandafter\def\csname PY@tok@gu\endcsname{\let\PY@bf=\textbf\def\PY@tc##1{\textcolor[rgb]{0.50,0.00,0.50}{##1}}}
\expandafter\def\csname PY@tok@s1\endcsname{\def\PY@tc##1{\textcolor[rgb]{0.73,0.13,0.13}{##1}}}
\expandafter\def\csname PY@tok@sb\endcsname{\def\PY@tc##1{\textcolor[rgb]{0.73,0.13,0.13}{##1}}}
\expandafter\def\csname PY@tok@nn\endcsname{\let\PY@bf=\textbf\def\PY@tc##1{\textcolor[rgb]{0.00,0.00,1.00}{##1}}}
\expandafter\def\csname PY@tok@cs\endcsname{\let\PY@it=\textit\def\PY@tc##1{\textcolor[rgb]{0.25,0.50,0.50}{##1}}}
\expandafter\def\csname PY@tok@nb\endcsname{\def\PY@tc##1{\textcolor[rgb]{0.00,0.50,0.00}{##1}}}
\expandafter\def\csname PY@tok@kd\endcsname{\let\PY@bf=\textbf\def\PY@tc##1{\textcolor[rgb]{0.00,0.50,0.00}{##1}}}
\expandafter\def\csname PY@tok@sa\endcsname{\def\PY@tc##1{\textcolor[rgb]{0.73,0.13,0.13}{##1}}}
\expandafter\def\csname PY@tok@nt\endcsname{\let\PY@bf=\textbf\def\PY@tc##1{\textcolor[rgb]{0.00,0.50,0.00}{##1}}}
\expandafter\def\csname PY@tok@gd\endcsname{\def\PY@tc##1{\textcolor[rgb]{0.63,0.00,0.00}{##1}}}
\expandafter\def\csname PY@tok@ss\endcsname{\def\PY@tc##1{\textcolor[rgb]{0.10,0.09,0.49}{##1}}}
\expandafter\def\csname PY@tok@w\endcsname{\def\PY@tc##1{\textcolor[rgb]{0.73,0.73,0.73}{##1}}}
\expandafter\def\csname PY@tok@vg\endcsname{\def\PY@tc##1{\textcolor[rgb]{0.10,0.09,0.49}{##1}}}
\expandafter\def\csname PY@tok@mo\endcsname{\def\PY@tc##1{\textcolor[rgb]{0.40,0.40,0.40}{##1}}}
\expandafter\def\csname PY@tok@mf\endcsname{\def\PY@tc##1{\textcolor[rgb]{0.40,0.40,0.40}{##1}}}
\expandafter\def\csname PY@tok@go\endcsname{\def\PY@tc##1{\textcolor[rgb]{0.53,0.53,0.53}{##1}}}
\expandafter\def\csname PY@tok@mh\endcsname{\def\PY@tc##1{\textcolor[rgb]{0.40,0.40,0.40}{##1}}}
\expandafter\def\csname PY@tok@dl\endcsname{\def\PY@tc##1{\textcolor[rgb]{0.73,0.13,0.13}{##1}}}
\expandafter\def\csname PY@tok@gi\endcsname{\def\PY@tc##1{\textcolor[rgb]{0.00,0.63,0.00}{##1}}}
\expandafter\def\csname PY@tok@kn\endcsname{\let\PY@bf=\textbf\def\PY@tc##1{\textcolor[rgb]{0.00,0.50,0.00}{##1}}}
\expandafter\def\csname PY@tok@vc\endcsname{\def\PY@tc##1{\textcolor[rgb]{0.10,0.09,0.49}{##1}}}
\expandafter\def\csname PY@tok@ni\endcsname{\let\PY@bf=\textbf\def\PY@tc##1{\textcolor[rgb]{0.60,0.60,0.60}{##1}}}
\expandafter\def\csname PY@tok@ch\endcsname{\let\PY@it=\textit\def\PY@tc##1{\textcolor[rgb]{0.25,0.50,0.50}{##1}}}
\expandafter\def\csname PY@tok@cpf\endcsname{\let\PY@it=\textit\def\PY@tc##1{\textcolor[rgb]{0.25,0.50,0.50}{##1}}}
\expandafter\def\csname PY@tok@sr\endcsname{\def\PY@tc##1{\textcolor[rgb]{0.73,0.40,0.53}{##1}}}
\expandafter\def\csname PY@tok@kt\endcsname{\def\PY@tc##1{\textcolor[rgb]{0.69,0.00,0.25}{##1}}}

\def\PYZbs{\char`\\}
\def\PYZus{\char`\_}
\def\PYZob{\char`\{}
\def\PYZcb{\char`\}}
\def\PYZca{\char`\^}
\def\PYZam{\char`\&}
\def\PYZlt{\char`\<}
\def\PYZgt{\char`\>}
\def\PYZsh{\char`\#}
\def\PYZpc{\char`\%}
\def\PYZdl{\char`\$}
\def\PYZhy{\char`\-}
\def\PYZsq{\char`\'}
\def\PYZdq{\char`\"}
\def\PYZti{\char`\~}
% for compatibility with earlier versions
\def\PYZat{@}
\def\PYZlb{[}
\def\PYZrb{]}
\makeatother


    % Exact colors from NB
    \definecolor{incolor}{rgb}{0.0, 0.0, 0.5}
    \definecolor{outcolor}{rgb}{0.545, 0.0, 0.0}



    
    % Prevent overflowing lines due to hard-to-break entities
    \sloppy 
    % Setup hyperref package
    \hypersetup{
      breaklinks=true,  % so long urls are correctly broken across lines
      colorlinks=true,
      urlcolor=urlcolor,
      linkcolor=linkcolor,
      citecolor=citecolor,
      }
    % Slightly bigger margins than the latex defaults
    
    \geometry{verbose,tmargin=1in,bmargin=1in,lmargin=1in,rmargin=1in}
    
    

    \begin{document}
    
    
    \maketitle
    
    

    
    \section{Formulas matemáticas (con
MathJax)}\label{formulas-matemuxe1ticas-con-mathjax}

    \subsection{Fórmulas en línea o en
bloque}\label{fuxf3rmulas-en-luxednea-o-en-bloque}

Tenemos dos formas de incluir formulas matemáticas, en medio del texto o
en forma de bloque. Para las formulas incuidas dentro de un texto,
llamadas en línea o \emph{inline}, se usa \textbf{un caracter
\texttt{\$}} antes y después de la fórmula. Para un bloque, se usan
\textbf{dos caracteres \texttt{\$} }.

Veámos el siguiente ejemplo:

\begin{quote}
La identidad de Euler, \(e^{i\pi} + 1 = 0\) es un caso especial de la
fórmula desarrollada por Leonhard Euler, notable por relacionar cinco
números muy utilizados en la historia de las matemáticas y que
pertenecen a distintas ramas de la misma
\end{quote}

La expresion \(e^{i\pi} + 1 = 0\) se representa así:
\texttt{\$e\^{}\{i\textbackslash{}pi\}\ +\ 1\ =\ 0\$.}

Las formulas de bloque se construyen con dos signos de dolar
\texttt{\$\$}, como en el siguiente ejemplo

\begin{quote}
Sea \(x\) un número real, la función logaritmo le asigna el exponente o
potencia \(n\) a la que un número fijo \(b\), llamado \emph{base}, se ha
de elevar para obtener dicho argumento. Es la función inversa de \(b\) a
la potencia \(n\). Esta función se escribe como:
\end{quote}

\begin{quote}
\[ \log_2 x = n \quad \iff \quad x = n^2 \]
\end{quote}

Si se accede al contenido de la celda, vemos que la expresión de
equivalencia se ha redactado así:
\texttt{\$\$\ \textbackslash{}log\_b\ x\ =\ n\ \textbackslash{}quad\ \textbackslash{}iff\ \textbackslash{}quad\ x\ =\ n\^{}2\ \$\$}.

    \subsection{Letras griegas}\label{letras-griegas}

Podemos usar letras griegas en minúsculas con
\texttt{\textbackslash{}alpha}, \texttt{\textbackslash{}beta},
\texttt{\textbackslash{}gamma}, ... \texttt{\textbackslash{}omega}:

\[ alfa: \alpha \] \[ beta: \beta \] \[ gamma: \gamma \] \[ ... \]
\[ omega: \omega \]

Y en mayúsculas con \texttt{\textbackslash{}Gamma},
\texttt{\textbackslash{}Delta} ... \texttt{\textbackslash{}Omega}:

\[ Gamma: \Gamma \] \[ Delta: \Delta \] \[ ... \] \[ Omega: \Omega \]

\begin{quote}
Las letras alfa mayúscula y beta mayúscula no se incluyen porque en
estos dos casos las formas gráficas son iguales que las letras
mayúyculas \texttt{A} y \texttt{B} latinas. Pasa lo mismo con épsilon
\(\epsilon\), micro \(\mu\), etc.
\end{quote}

    \subsection{Símbolos matemáticos
habituales}\label{suxedmbolos-matemuxe1ticos-habituales}

Algunos de los símbolos más habituales pueden verse en esta tabla:

\begin{longtable}[c]{@{}llllll@{}}
\toprule
Símbolo & Código Latex & Símbolo & Código Latex & Símbolo & Código
Latex\tabularnewline
\midrule
\endhead
\(\neq\) & \texttt{\textbackslash{}neq} & \(\pm\) &
\texttt{\textbackslash{}pm} & \(\gets\) &
\texttt{\textbackslash{}gets}\tabularnewline
\(\impliedby\) & \texttt{\textbackslash{}impliedby} & \(\implies\) &
\texttt{\textbackslash{}implies} & \(\to\) &
\texttt{\textbackslash{}to}\tabularnewline
\(\leqslant\) & \texttt{\textbackslash{}leqslant} & \(\mp\) &
\texttt{\textbackslash{}mp} & \(\iff\) &
\texttt{\textbackslash{}iff}\tabularnewline
\(\geqslant\) & \texttt{\textbackslash{}leqslant} & \(\times\) &
\texttt{\textbackslash{}times} & \(\$\) &
\texttt{\textbackslash{}\$}\tabularnewline
\(\approx\) & \texttt{\textbackslash{}approx} & \(\div\) &
\texttt{\textbackslash{}div} & \(\wr\) &
\texttt{\textbackslash{}wr}\tabularnewline
\(\equiv\) & \texttt{\textbackslash{}equiv} & \(\cup\) &
\texttt{\textbackslash{}cup} & \(\cap\) &
\texttt{\textbackslash{}cap}\tabularnewline
\(\cong\) & \texttt{\textbackslash{}cong} & \(\simeq\) &
\texttt{\textbackslash{}simeq} & \(\{\) &
\texttt{\textbackslash{}\{}\tabularnewline
\(\in\) & \texttt{\textbackslash{}in} & \(\notin\) &
\texttt{\textbackslash{}notin} & \(\}\) &
\texttt{\textbackslash{}\}}\tabularnewline
\(\partial\) & \texttt{\textbackslash{}partial} & \(\infty\) &
\texttt{\textbackslash{}infty} & \(\ell\) &
\texttt{\textbackslash{}ell}\tabularnewline
\(\subset\) & \texttt{\textbackslash{}subset} & \(\supset\) &
\texttt{\textbackslash{}supset} & \(\varnothing\) &
\texttt{\textbackslash{}varnothing}\tabularnewline
\(\subseteq\) & \texttt{\textbackslash{}subseteq} & \(\supseteq\) &
\texttt{\textbackslash{}supseteq} & \(\S\) &
\texttt{\textbackslash{}S}\tabularnewline
\(\vee\) & \texttt{\textbackslash{}vee} & \(\wedge\) &
\texttt{\textbackslash{}wedge} & \(\cdot\) &
\texttt{\textbackslash{}cdot}\tabularnewline
\(\forall\) & \texttt{\textbackslash{}forall} & \(\exists\) &
\texttt{\textbackslash{}exists} & \(\ast\) &
\texttt{\textbackslash{}ast}\tabularnewline
\(\checkmark\) & \texttt{\textbackslash{}checkmark} & \(\nabla\) &
\texttt{\textbackslash{}nabla} & \(\aleph\) &
\texttt{\textbackslash{}aleph}\tabularnewline
\bottomrule
\end{longtable}

    \subsection{Superíndices y
subíndices}\label{superuxedndices-y-subuxedndices}

Para los superíndices se usa el caracter \texttt{\^{}} y para subíndices
\texttt{\_}. Por ejemplo \texttt{x\_i\^{}2} se vería así:

\[ x_i^2 \]

y \texttt{\textbackslash{}log\_2\ x} así:

\[ \log_2 x \]

    \subsection{Grupos}\label{grupos}

Los superíndices, los subíndices y otras operaciones que veremos sólo se
aplican al siguiente \emph{grupo}, entendiendo por grupo o bien un solo
caracter o una expresión entre llaves (\{ ... \}). POr ejemplo, si
queremos expresar la potencia vigesimo cuarta de 2, la expresión
\texttt{2\^{}24} no nos dará lo esperado:

\[ 2^24 \]

Para conseguir lo que queremos hay que expresar el exponente como un
grupo con llaves, es decir \texttt{2\^{}\{24\}}:

\[ 2^{24} \]

También se pueden añadir llaves para delimitar el grupo al que el
subíndice o superíndice debe aplicarse, por ejemplo
\texttt{\{x\^{}y\}\^{}z}:

\[{x^y}^z \]

es distinto de \texttt{x\^{}\{y\^{}z\}}{]}:

\[x^{y^z} \]

Observese este otro ejemplo, la diferencia entre \texttt{x\_i\^{}2} y
\texttt{x\_\{i\^{}2\}}:

\[ x_i^2 \] \[ x_{i^2} \]

    \subsection{Paréntesis}\label{paruxe9ntesis}

Se pueden usar sin problemas los signos de paréntesis: \texttt{(} y
\texttt{)} y corchetes: \texttt{{[}} y \texttt{{]}}, pero para usar las
llaves de forma literal hay que escaparlas: '\{' y '\}'.

El problema es que estos caracteres no escalan proporcionalmente a la
expresión que haya en su interior, así que si se escribe
\texttt{(\textbackslash{}frac\{\textbackslash{}sqrt\ x\}\{y\^{}3\})},
los paréntesis serán demasiado pequeños:

\[ (\frac{\sqrt x}{y^3}) \]

Para evitarlo se pueden usar las expresiones
\texttt{\textbackslash{}left(} y \texttt{\textbackslash{}right)}, que si
se adaptan dinámicamente a su contenido:

\[ \left(\frac{\sqrt x}{y^3}\right) \]

También podemos usar a los corchetes (\texttt{\textbackslash{}left{[}} y
\texttt{\textbackslash{}right{]}}):

\[ \left[ \left(\frac{\sqrt x}{y^3}\right) \alpha^2 \right] \]

    \subsection{Sumatorios e integrales}\label{sumatorios-e-integrales}

Para obtener expresiones de sumatorios e integrales, se usan
respectivamente \texttt{\textbackslash{}sum} e
\texttt{\textbackslash{}int}. El subíndice será el límite inferior y el
superíndice el límite superior. Por ejemplo:
\texttt{\textbackslash{}sum\_1\^{}n} equivale a:

\[ \sum_1^n \]

Como vimos antes, hay que agrupar con llaves si los límites contienen
más de un símbolo, por ejemplo
\texttt{\textbackslash{}int\_\{i=0\}\^{}\textbackslash{}infty\ i\^{}2}
sería:

\[ \int_{i=0}^\infty i^2 \]

Otras posibilidades son \texttt{\textbackslash{}prod} \(\prod\),
\texttt{\textbackslash{}bigcup} \(\bigcup\),
\texttt{\textbackslash{}bigcap} \(\bigcap\) e \(`\iint`\).

    \subsection{Fracciones}\label{fracciones}

Hay dos formas de representar fracciones:
\texttt{\textbackslash{}frac\ ab} se aplica a los dos grupos siguientes:

\[\frac ab\]

Si las espresiones del numerador o denominador son más complejas, las
podemos agrupar con las llaves, por ejemplo
\texttt{\textbackslash{}frac\{a+1\}\{b-1\}} resulta en:

\[ \frac {a+1} {b-1} \]

La otra forma es habitual para expresiones más complicadas, usando
\texttt{\textbackslash{}over} se puede dividir un grupo para formar la
fracción, por ejemplo, la expresión anterior con over sería
\texttt{\{a+1\textbackslash{}over\ b-1\}}, que se vería:

\[ {a+1\over b-1} \]

    \subsection{\texorpdfstring{Tipografías
(\emph{fonts})}{Tipografías (fonts)}}\label{tipografuxedas-fonts}

Existen varias opciones:

\begin{itemize}
\item
  Se puede usar \texttt{\textbackslash{}mathbb} o
  \texttt{\textbackslash{}Bbb} para el estilo \emph{pizarra clásica}:
  \(\mathbb{ABC\ldots WXYZ}\)
\item
  \texttt{\textbackslash{}mathbf} para tipografia \emph{bold} o negrita:
  \(\mathbf{ABC\ldots WXYZ}\)
\item
  \texttt{\textbackslash{}mathtt} para tipografía monoespaciada:
  \(\mathtt{ABC\ldots WXYZ}\)
\item
  \texttt{\textbackslash{}mathfrak} para tipografía \emph{Fraktur} a
  Germánica: \(\mathfrak{ABC\ldots WXYZ}\)
\item
  \texttt{\textbackslash{}mathrm} para tipografía Roman:
  \(\mathrm{ABC\ldots WXYZ}\)
\item
  \texttt{\textbackslash{}mathsf} para tipografía \emph{sans-serif}:
  \(\mathsf{ABC\ldots WXYZ}\)
\item
  \texttt{\textbackslash{}mathcal} para tipografía caligráfica:
  \(\mathcal{ABC\ldots WXYZ}\)
\item
  \texttt{\textbackslash{}mathscr} para tipografía manuscrita o
  \emph{script}: \(\mathscr{ABC\ldots WXYZ}\)
\end{itemize}

    \subsection{Raíces}\label{rauxedces}

Usando \texttt{\textbackslash{}sqrt} podemos ajustarnos al contenido,
como vimos con los paréntesis antes. Con
\texttt{\textbackslash{}sqrt\{x\^{}3\}} obtenemos:

\[ \sqrt{x^3} \]

Para raices distintas de la raíz cuadrada, usamos corchetes para indicar
el índice de la raíz, es decir, para la raiz cúbica
\texttt{\textbackslash{}sqrt{[}3{]}\{\textbackslash{}frac\ \{x\^{}2\}y\}}
genera:

\[ \sqrt[3]{\frac {x^2}y} \]

Para expresiones complejas, se puede usar la forma
\texttt{\{...\}\^{}\{1/2\}}, por ejemplo:

\[ {\left(x \sin\phi + z \cos\phi\right)}^{1/2} \]

    \subsection{Funciones especiales}\label{funciones-especiales}

Algunas funciones especiales como lim, sin, max, etc... se escribe
normalmente con tipografia románica en vez de itálica. Para conseguirlo
hay que usar \texttt{\textbackslash{}lim}, \texttt{\textbackslash{}sin},
\texttt{\textbackslash{}max}, etc... Véase la diferencia entre
\texttt{sin(x)} y \texttt{\textbackslash{}sin(x)}:

\[ sin(x)  \quad | \quad \sin(x) \]

Para los límites, se usa un subíndice para incluir una anotación, como
en \texttt{\textbackslash{}lim\_\{x\textbackslash{}to\ 0\}}:

\[ \lim_{x\to 0} \]

    \subsection{Acentos especiales}\label{acentos-especiales}

Hay una serie de marcas para añadir acentos especiales a caracteres
normales. Algunos de los más usados son:

\begin{longtable}[c]{@{}ll@{}}
\toprule
Codigo & resultado\tabularnewline
\midrule
\endhead
\texttt{\textbackslash{}tilde\{a\}} & \(\tilde{a}\)\tabularnewline
\texttt{\textbackslash{}hat\{a\}} & \(\hat{a}\)\tabularnewline
\texttt{\textbackslash{}check\{a\}} & \(\check{a}\)\tabularnewline
\texttt{\textbackslash{}vec\{a\}} & \(\vec{a}\)\tabularnewline
\texttt{\textbackslash{}bar\{a\}} & \(\bar{a}\)\tabularnewline
\texttt{\textbackslash{}acute\{a\}} & \(\acute{a}\)\tabularnewline
\texttt{\textbackslash{}grave\{a\}} & \(\grave{a}\)\tabularnewline
\texttt{\textbackslash{}breve\{a\}} & \(\breve{a}\)\tabularnewline
\texttt{\textbackslash{}dot\{a\}} & \(\dot{a}\)\tabularnewline
\texttt{\textbackslash{}ddot\{a\}} & \(\ddot{a}\)\tabularnewline
\texttt{\textbackslash{}dddot\{a\}} & \(\dddot{a}\)\tabularnewline
\texttt{\textbackslash{}ddddot\{a\}} & \(\ddddot{a}\)\tabularnewline
\texttt{\textbackslash{}mathring\{a\}} & \(\mathring{a}\)\tabularnewline
\texttt{\textbackslash{}boxed\{a\}} & \(\boxed{a}\)\tabularnewline
\bottomrule
\end{longtable}

En la tabla se muestran con la letra \texttt{a}, pero se pueden aplicar
a cualquier letra

    \subsection{Puntos suspensivos}\label{puntos-suspensivos}

Se puede user \texttt{\textbackslash{}ldots} para obtener puntos como en

\[ a_1, a_2, \ldots, a_n \]

O usar \texttt{\textbackslash{}cdots} para obtener:

\[ a_1+a_2+\cdots+a_n \]

    \subsection{Resaltar ecuaciones}\label{resaltar-ecuaciones}

Si queremos resaltar una ecuación podemos usar
\texttt{\textbackslash{}dbox}, de forma que:

\begin{verbatim}
$$ \bbox[yellow]
{
e^x=\lim_{n\to\infty} \left( 1+\frac{x}{n} \right)^n \qquad (1)
}
$$
\end{verbatim}

Produce:

\[ \bbox[yellow,5px]
{
e^x=\lim_{n\to\infty} \left( 1+\frac{x}{n} \right)^n \qquad (1)
}
\]

    \subsection{Matrices}\label{matrices}

Usando
\texttt{\textbackslash{}begin\{matrix\}\ ...\ \textbackslash{}end\{matrix\}}
podemos incluir una matriz. Para separar las columnas se debe usar el
caracter \texttt{\&} y para separar las filas se usan dos barras
invertiddas \texttt{\textbackslash{}\textbackslash{}}. Por ejemplo:

\begin{verbatim}
\begin{matrix}
1 & x & x^2 \\
1 & y^2 & y^2 \\
1 & z & z^2 \\
\end{matrix}
\end{verbatim}

Produce:

\[
    \begin{matrix}
    1 & x & x^2 \\
    1 & y^2 & y^2 \\
    1 & z & z^2 \\
    \end{matrix}
\]

Los espacios se ajustaran de forma automática. Si queremos podemos
añadir paréntesis, corchetes o llaves usando las expresiones que vimos
en un apartado anterior, o usando, en vez de \texttt{matrix},
\texttt{pmatrix} para paréntesis, \texttt{bmatrix} para corchetes,
\texttt{Bmatrix} para llaves, \texttt{vmatrix} para barras y
\texttt{Vmatrix} para doble barra.

Puedes usar \texttt{\textbackslash{}cdots} \(\cdots\),
\texttt{\textbackslash{}ddots} \(\ddots\) y
\texttt{\textbackslash{}vdots} \(\vdots\) para omitir partes de la
matriz.

\[
\begin{vmatrix}
1      & a_1    & a_1^2  & \cdots & a_1^n  \\
1      & a_2    & a_2^2  & \cdots & a_2^n  \\
\vdots & \vdots & \vdots & \ddots & \vdots \\
1      & a_m    & a_m^2  & \cdots & a_m^n  \\
\end{vmatrix}
\]

    \subsection{Incluir Latex}\label{incluir-latex}

Podemos usar el comando magico \%\%latex para incluir código Latex puro

    \begin{Verbatim}[commandchars=\\\{\}]
{\color{incolor}In [{\color{incolor}3}]:} \PY{c}{\PYZpc{}\PYZpc{}latex}
        \PY{k}{\PYZbs{}begin}\PY{n+nb}{\PYZob{}}align\PY{n+nb}{\PYZcb{}}
        \PY{k}{\PYZbs{}nabla} \PY{k}{\PYZbs{}times} \PY{k}{\PYZbs{}vec}\PY{n+nb}{\PYZob{}}\PY{k}{\PYZbs{}mathbf}\PY{n+nb}{\PYZob{}}B\PY{n+nb}{\PYZcb{}}\PY{n+nb}{\PYZcb{}} \PYZhy{}\PY{k}{\PYZbs{},} \PY{k}{\PYZbs{}frac}1c\PY{k}{\PYZbs{},} \PY{k}{\PYZbs{}frac}\PY{n+nb}{\PYZob{}}\PY{k}{\PYZbs{}partial}\PY{k}{\PYZbs{}vec}\PY{n+nb}{\PYZob{}}\PY{k}{\PYZbs{}mathbf}\PY{n+nb}{\PYZob{}}E\PY{n+nb}{\PYZcb{}}\PY{n+nb}{\PYZcb{}}\PY{n+nb}{\PYZcb{}}\PY{n+nb}{\PYZob{}}\PY{k}{\PYZbs{}partial} t\PY{n+nb}{\PYZcb{}} \PY{n+nb}{\PYZam{}} = \PY{k}{\PYZbs{}frac}\PY{n+nb}{\PYZob{}}4\PY{k}{\PYZbs{}pi}\PY{n+nb}{\PYZcb{}}\PY{n+nb}{\PYZob{}}c\PY{n+nb}{\PYZcb{}}\PY{k}{\PYZbs{}vec}\PY{n+nb}{\PYZob{}}\PY{k}{\PYZbs{}mathbf}\PY{n+nb}{\PYZob{}}j\PY{n+nb}{\PYZcb{}}\PY{n+nb}{\PYZcb{}} \PY{k}{\PYZbs{}\PYZbs{}}
        \PY{k}{\PYZbs{}nabla} \PY{k}{\PYZbs{}cdot} \PY{k}{\PYZbs{}vec}\PY{n+nb}{\PYZob{}}\PY{k}{\PYZbs{}mathbf}\PY{n+nb}{\PYZob{}}E\PY{n+nb}{\PYZcb{}}\PY{n+nb}{\PYZcb{}} \PY{n+nb}{\PYZam{}} = 4 \PY{k}{\PYZbs{}pi} \PY{k}{\PYZbs{}rho} \PY{k}{\PYZbs{}\PYZbs{}}
        \PY{k}{\PYZbs{}nabla} \PY{k}{\PYZbs{}times} \PY{k}{\PYZbs{}vec}\PY{n+nb}{\PYZob{}}\PY{k}{\PYZbs{}mathbf}\PY{n+nb}{\PYZob{}}E\PY{n+nb}{\PYZcb{}}\PY{n+nb}{\PYZcb{}}\PY{k}{\PYZbs{},} +\PY{k}{\PYZbs{},} \PY{k}{\PYZbs{}frac}1c\PY{k}{\PYZbs{},} \PY{k}{\PYZbs{}frac}\PY{n+nb}{\PYZob{}}\PY{k}{\PYZbs{}partial}\PY{k}{\PYZbs{}vec}\PY{n+nb}{\PYZob{}}\PY{k}{\PYZbs{}mathbf}\PY{n+nb}{\PYZob{}}B\PY{n+nb}{\PYZcb{}}\PY{n+nb}{\PYZcb{}}\PY{n+nb}{\PYZcb{}}\PY{n+nb}{\PYZob{}}\PY{k}{\PYZbs{}partial} t\PY{n+nb}{\PYZcb{}} \PY{n+nb}{\PYZam{}} = \PY{k}{\PYZbs{}vec}\PY{n+nb}{\PYZob{}}\PY{k}{\PYZbs{}mathbf}\PY{n+nb}{\PYZob{}}0\PY{n+nb}{\PYZcb{}}\PY{n+nb}{\PYZcb{}} \PY{k}{\PYZbs{}\PYZbs{}}
        \PY{k}{\PYZbs{}nabla} \PY{k}{\PYZbs{}cdot} \PY{k}{\PYZbs{}vec}\PY{n+nb}{\PYZob{}}\PY{k}{\PYZbs{}mathbf}\PY{n+nb}{\PYZob{}}B\PY{n+nb}{\PYZcb{}}\PY{n+nb}{\PYZcb{}} \PY{n+nb}{\PYZam{}} = 0
        \PY{k}{\PYZbs{}end}\PY{n+nb}{\PYZob{}}align\PY{n+nb}{\PYZcb{}}
\end{Verbatim}


    \begin{align}
\nabla \times \vec{\mathbf{B}} -\, \frac1c\, \frac{\partial\vec{\mathbf{E}}}{\partial t} & = \frac{4\pi}{c}\vec{\mathbf{j}} \\
\nabla \cdot \vec{\mathbf{E}} & = 4 \pi \rho \\
\nabla \times \vec{\mathbf{E}}\, +\, \frac1c\, \frac{\partial\vec{\mathbf{B}}}{\partial t} & = \vec{\mathbf{0}} \\
\nabla \cdot \vec{\mathbf{B}} & = 0
\end{align}

    
    \subsection{Referencias y recursos}\label{referencias-y-recursos}

\begin{itemize}
\item
  Este resumen sacada de Stack Overflow
  \href{https://math.meta.stackexchange.com/questionsa/5020/mathjax-basic-tutorial-and-quick-reference}{Mathjax
  basic tutorial and quick reference}
\item
  \href{http://jupyter-notebook.readthedocs.io/en/stable/examples/Notebook/Typesetting\%20Equations.html}{Ejemplos
  de la documentación de Jupyter}
\item
  \href{http://detexify.kirelabs.org/classify.html}{Detexify} te permite
  dibujar un símbolo en una página web y te muestra los símbolos TeX que
  más se le parecen. No obstante, Mathjax no soporta todo el conjunto de
  caracteres que soporta TeX. Pero es un buen punto para empezar.
\item
  MathJax.org mantienen
  \href{http://docs.mathjax.org/en/latest/tex.html\#supported-latex-commands}{una
  lista de ordenes LATEX soportadas}.
\item
  La página de la doctora Carol J.V. FISHER BURNS sobre
  \href{http://www.onemathematicalcat.org/MathJaxDocumentation/TeXSyntax.htm}{ordenes
  tex disponibles en MathJax} es muy completa.
\item
  Latex tiene muchos más símbolos dispobibles; se puede consultar un
  listado abreviado aqui: \url{http://pic.plover.com/MISC/symbols.pdf}
\item
  Y otros mucho más extenso en:
  \url{http://library.caltech.edu/etd/symbols-a4.pdf}
\end{itemize}


    % Add a bibliography block to the postdoc
    
    
    
    \end{document}
